\documentclass[a4paper, 11pt, titlepage]{article}

\usepackage{minted}
\usepackage[utf8]{inputenc}
\usepackage[T1]{fontenc}
\usepackage{amsmath, amsfonts, amssymb}
\usepackage{hyperref}

\author{Kévin \textsc{Le Run}, Théophile \textsc{Bastian}, Rémi \textsc{Oudin}}
\title{Installing Humanoid Path Planner}
\date{24\textsuperscript{th}; 2017}

\begin{document}

\maketitle

\tableofcontents

\newpage

\section*{Introduction}
\label{sec:intro}
\addcontentsline{toc}{section}{\nameref{sec:intro}}

This short report explaoins our attempts to install Humanoid Path Planner,
a library created by the \url{http://projects.laas.fr/gepetto/index.php}\\

This library has been developped in the Laas since 2013, and has been designed
in order to model humanoid robot movements. It is also easy to create usable
images for robots, since it is strongly interfaced with
\url{http://www.ros.org/} (Robot Operating System).\\

We have experienced multiple issues while installing HPP, and even after a
``uccesful'' installation, the library doesn't behave well.

In this short report, we will show multiple flaws in the installation process,
and what could improve the installation.

\section{First step: Installing ROS}

\subsection{A too strict installer}
ROS is an operating system for robots, and it is a mandatory dependency of hpp,
even though we do not want to create bootable images for robots.

Here comes the first issue with the installation of HPP~: It forces to use one
precise version of ROS\@: \verb|ros-indigo|.

This version is currently the OldOldStable version of ROS\@. Thus, it has
disappeared from many mirrors and Linux distribution, or depends from many
packages that have disappeared from Linux distribution.

For now, the only distribution on which we had a ``uccesful'' installation were
Xubuntu 14.04 and Ubuntu 12.04.

On more recent Linux distribution, \verb|ros-indigo| has disappeared from the
repositories. Only \verb|ros-jade| and \verb|ros-kinetic| are available.

Many \verb|ros| distribution are retro-compatible, and thus, the installer
should check for the latest version available, instead of crashing.

Second issue, if \verb|ros| is not installed in a precise directory
(\verb|/opt/ros-indigo/|) the installation crashes.

\subsection{Our attempts to install ROS-indigo}

\subsubsection{On our devices}

    Our first try was on our devices (Arch LInux and Debian Stretch)
    ROS provides a ``ackage manager'', which seems to be still quite
    experimental, and didn't manage to install \verb|ros|, since relies on the
    distribution of the devices and was unable to install some packages, for
    versioning reasons.

\subsubsection{On dedicated servers}

    Then, we created a first server, using Debian Jessie, which is supposed to
    be compatible with ros-indigo.

    Some packages versions had changed, and the installation of ros-indigo failed.

    Trying to use \verb|catkin|, another \verb|ros| package manager, the
    installation still failed, for unknown reasons.

    We finally created a server with Ubuntu 12.04, the old old stable version of
    Ubuntu, however still in LTS\@.

    We eventually managed to install ros-indigo.

\section{Installing HPP}

In order to install hpp, we followed the instructions on this webpage
\url{https://humanoid-path-planner.github.io/hpp-doc/download.html?branch=master}

The dependency building of \verb|openscenegraph| failed, but an installation
still does the job, even though it should be unnecessary.
\end{document}
