\documentclass[11pt]{beamer}
\usetheme{Warsaw}
\usepackage[utf8]{inputenc}
\usepackage[french]{babel}
\usepackage[T1]{fontenc}
\usepackage{amsmath}
\usepackage{amsfonts}
\usepackage{amssymb}
\usepackage{graphicx}
\usepackage{tikz}
\setbeamertemplate{navigation symbols}{}

%%%%%%%%%%%%%%%%%%%%%%%%%%%%%%%%%%%%%%%%%%%%%%%%%%%%%%%%%
\author{Théophile \textsc{Bastian}, Kévin \textsc{Le Run}, Rémi \textsc{Oudin}}
\title{HPP \& \og{}Hilare pulling a trailer~\fg}
\date{24 janvier 2017}
%\subject{}
%\logo{}
%\institute{}
\begin{document}

\begin{frame}
	\titlepage{}
	\tableofcontents
\end{frame}

%\begin{frame}
%\tableofcontents
%\end{frame}

\section{Troubles installing Humanoid Path Planner (HPP)}

\subsection{Installing Robot Operating System (ROS)}

\begin{frame}
    \frametitle{\subsecname}
    \begin{itemize}
        \item A part of Ros-indigo is a mandatory dependency of HPP.\\
        \item Indigo is currently the \emph{Old Old Stable} version of ROS\@.
        \item It supports only Ubuntu 12.04 and Ubuntu 14.04
        \item Any other distributions are either experimental or too modern.
        \item It does not provide support for recent versions of Debian
        \item Arch installation was also broken
    \end{itemize}
\end{frame}

\subsubsection{Solutions?}

\begin{frame}
    \frametitle{\subsecname}
    \begin{enumerate}
        \item Tried on our computers.
        \item Created a server with Debian Jessie. Installation of ROS worked,
            but hpp failed
        \item Created an Ubuntu 12.04 Server. Installation of ROS worked,
            and hpp eventually succeeded.
    \end{enumerate}
\end{frame}

\section{HPP Architecture}

\begin{frame}
    \begin{figure}
        \begin{center}
            \begin{tikzpicture}
                \node at (0,1)
                {\includegraphics[width=0.1\textwidth]{server_icon.png}};
                \node[rectangle, draw] (c) at (0,0) {CORBA};
                \onslide<2->{%
                    \node[rectangle, draw] (r) at (4,2) {Robots};
                    \node[rectangle, draw] (o) at (4,0) {Obstacles};
                    \node[rectangle, draw] (p) at (4,-2) {Problèmes};
                    \draw[->] (c) edge (r) (c) edge (o) (c) edge (p);
                }
                \onslide<3->{%
                    \node[rectangle, draw] (c1) at (-4,1) {Client};
                    \node at (-4,0) {$\vdots$};
                    \node[rectangle, draw] (c2) at (-4,-1) {Client};
                    \draw[->] (c1) edge (c) (c2) edge (c);
                }
            \end{tikzpicture}
        \end{center}
    \end{figure}
\end{frame}
\end{document}
