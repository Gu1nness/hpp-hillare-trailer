% vim: :spell spelllang=fr

\documentclass[11pt]{beamer}
\usetheme{Warsaw}
\usepackage[utf8]{inputenc}
\usepackage[french]{babel}
\usepackage[T1]{fontenc}
\usepackage{amsmath}
\usepackage{amsfonts}
\usepackage{amssymb}
\usepackage{graphicx}
\usepackage{tikz}
\setbeamertemplate{navigation symbols}{}

\newcommand{\abs}[1]{\left\vert{} #1 \right\vert{}}
\newcommand{\steer}{\textbf{Dirige}}
\newcommand{\steerflat}{\steer^\Delta}
\newcommand{\confspace}{\mathcal{C}}
\newcommand{\qcusp}{q_{\text{cassure}}}

%%%%%%%%%%%%%%%%%%%%%%%%%%%%%%%%%%%%%%%%%%%%%%%%%%%%%%%%%
\author{Théophile \textsc{Bastian}, Kévin \textsc{Le Run}, Rémi \textsc{Oudin}}
\title{HPP \& \og{}Hilare pulling a trailer~\fg}
\date{24 janvier 2017}
%\subject{}
%\logo{}
%\institute{}
\begin{document}

\begin{frame}
	\titlepage{}
	\tableofcontents
\end{frame}

%\begin{frame}
%\tableofcontents
%\end{frame}

% TODO intro : ce qu'on devait faire, pourquoi on l'a pas fait.

\section{Problèmes d'installation de HPP}

\subsection{Installation de ROS}

\begin{frame}
    \frametitle{\subsecname}
    \begin{itemize}
        \item Une partie de ROS indigo en dépendance obligatoire.
        \item Indigo est actuellement la version \emph{Old Old Stable} de ROS\@.
        \item Ne fonctionne que sous Ubuntu 12.04 et 14.04
        \item Toute autre distribution est soit expérimentale soit trop moderne
        \item Pas de support pour les versions récentes de Debian
        \item L'installation sous Arch et Xubuntu n'est pas non plus
            fonctionnelle
    \end{itemize}
\end{frame}

\subsubsection{Solutions~?}

\begin{frame}
    \frametitle{\subsecname}
    \begin{enumerate}
        \item Essayé nos machines
        \item Essayé sur un serveur Debian Jessie. ROS ok mais pas HPP\@.
        \item Essayé sur un serveur Ubuntu 12.04. ROS et HPP ok.
    \end{enumerate}
\end{frame}

\section{Architecture de HPP}

\begin{frame}
    \begin{figure}
        \begin{center}
            \begin{tikzpicture}
                \node at (0,1)
                {\includegraphics[width=0.1\textwidth]{server_icon.png}};
                \node[rectangle, draw] (c) at (0,0) {CORBA};
                \onslide<2->{%
                    \node[rectangle, draw] (r) at (4,2) {Robots};
                    \node[rectangle, draw] (o) at (4,0) {Obstacles};
                    \node[rectangle, draw] (p) at (4,-2) {Problèmes};
                    \draw[->] (c) edge (r) (c) edge (o) (c) edge (p);
                }
                \onslide<3->{%
                    \node[rectangle, draw] (c1) at (-4,1) {Client};
                    \node at (-4,0) {$\vdots$};
                    \node[rectangle, draw] (c2) at (-4,-1) {Client};
                    \draw[->] (c1) edge (c) (c2) edge (c);
                }
            \end{tikzpicture}
        \end{center}
    \end{figure}
\end{frame}

%%%%%%%% (Tobast) Hilare pulling a trailer %%%%%%%%%%%%%%%%%%%%%%%%%%%

\section{\og{}Hilare pulling a trailer~\fg}

\begin{frame}{Hilare pulling a trailer}
	\begin{centering}
		\textit{\Large Motion Planning and Control for Hilare Pulling a
		Trailer} \\
		\vspace{1em}
		{F. Lamiraux, S. Sekhavat, J. P. Laumond} \\
		\vspace{1em}
		1999 \\
	\end{centering}
\end{frame}

\begin{frame}{Problème à résoudre}
	\begin{itemize}
		\item Hilare~: robot à roues, vitesses indépendantes
		\item Remorque~: attache centrée sur l'axe des roues (A) ou à l'arrière
			(B)
		\item Objectif~: planifier une trajectoire (chemin réalisable)
		\item Système non-holonome
		\item Contraintes en vitesses/accélérations maximales
			\[ \abs{v_r} \leq v_{\max} \quad
			\abs{\omega_r} \leq \omega_{\max} \quad
			\abs{\dot{v_r}} \leq \dot{v}_{\max} \quad
			\abs{\dot{\omega_r}} \leq \dot{\omega}_{\max} \]
	\end{itemize}
\end{frame}

\begin{frame}{Vue d'ensemble}
	Algorithmes successifs~:
	\begin{itemize}
		\item trajectoire \alert{géométrique}, évite les obstacles~;
			\pause{}
		\item trajectoire \alert{cinématique}, prend en compte la
			non-holonomie~;
			\pause{}
		\item lissage~;
	\end{itemize}
	\hspace{2cm}\alert{$\implies$ chemin}
			\pause{}
	\begin{itemize}
		\item application des contraintes
	\end{itemize}
	\hspace{2cm}\alert{$\implies$ trajectoire}
\end{frame}

\begin{frame}{Trajectoire géométrique}
	Trajectoire \og{}macroscopique~\fg~: système considéré holonome
	\begin{itemize}
		\item Random Path Planner (Barraquand, Latombe, 1991)
		\item Attraction par \og{}potentiel~\fg{} vers l'objectif
		\item Si minimum local, mouvement brownien
		\item On itère
		\item On lisse
			\vspace{1em}
		\item probabilistement complet~: solution existe $\implies$
			$\mathbb{P}(\text{la trouver}) \rightarrow 1$ (temps recherche
			$\rightarrow \infty$)
	\end{itemize}
\end{frame}

\begin{frame}{Non-holonomie}
	\og{}Propriété topologique~\fg~:
	\begin{itemize}
		\item $\confspace$~: configurations $(x, y, \theta, \varphi)$
		\item $\steer: \confspace \times \confspace \to C^0(\left[0, 1\right],
			\confspace)$
		\item $\steer(q_i, q_f)(0) = q_i$
		\item $\steer(q_i, q_f)(1) = q_f$
		\item $\forall \varepsilon > 0, \forall \eta > 0$,
			\[d(q_i, q_f) < \eta \implies \forall t \in [0, 1],
			d(q_i, \steer(q_i, q_f)(t)) < \varepsilon \]
	\end{itemize}
\end{frame}

\begin{frame}{Fonction de direction}
	\begin{itemize}
		\item Voulue très lisse~!
		\item première idée~: courbe canonique $\gamma_q$, arc de cercle
			passant par $q$ conservant $\varphi$
		\item $\gamma(t) = \alpha(t) \gamma_{q_i}(vt) + (1-\alpha(t))
			\gamma_{q_f}(vt)$
		\item atteint un \og{}cône~\fg{}
		\item ne respecte pas la propriété topologique
	\end{itemize}
\end{frame}

\begin{frame}{Fonction de direction (cont.)}
	\begin{itemize}
		\item mais perturber $q_f$ modifie peu la courbe
		\item[~] $\implies$ point de cassure
		\item $d(q_i, q_f)$ faible $\leadsto$ chemin possible par
			$\qcusp$ en restant proche
		\item $\leadsto$ $\steerflat$
	\end{itemize}

	\vspace{-3em}
	\begin{figure}
		\flushright{}
		\includegraphics[width=0.5\textwidth]{img/steer.png}
	\end{figure}
\end{frame}

\begin{frame}{Chemin cinématique}
	\begin{itemize}
		\item Récursivement sur le chemin géométrique
		\item Si $\steerflat$ peut joindre les points (sans toucher
			d'obstactles), joindre
		\item Sinon, séparer en deux bouts et réappliquer
	\end{itemize}

	\vspace{1em}

	Phase de lissage~:

	\begin{itemize}
		\item Prendre aléatoirement des points d'arrêt
		\item Essayer de les joindre avec l'algorithme précédent
		\item Arrêt quand le nombre de points ne diminue plus
	\end{itemize}
\end{frame}

\begin{frame}{Trajectoire}
	\begin{itemize}
		\item Objectif~: trouver $s : \mathbb{R}_+ \to [0, 1]$ croissante
			(\og{}temps~\fg)
		\item $\forall t, q(t) = \gamma(s(t))$
		\item Représentation des contraintes sur les vitesses, accélérations
			dans le plan $(s, \dot{s})$
		\item Utilisation d'accélérations constantes par morceaux \og{}tant que
			possible~\fg{}
		\item Bon compromis optimalité-temps de calcul~: calcul embarqué sur
			Hilare
	\end{itemize}
\end{frame}

\end{document}
